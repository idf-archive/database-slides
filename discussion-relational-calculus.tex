%&latex

\documentclass{beamer}
\usepackage{verbatim}
\usefonttheme{serif}
% Customize slide appearance
\mode<presentation>
{
  \usetheme{Warsaw}
  \setbeamercovered{transparent}
}


\usepackage[english]{babel}
\usepackage{times}

% You can add any graphics to every slide by following command:
% \logo{\includegraphics{logo.eps}}

% Uncomment this, if you want the table of contents before each subsection.
% However, to edit slides in TeXWord avoiding this feature is good idea.
% \AtBeginSubsection[]
% {
%   \begin{frame}<beamer>
%     \frametitle{Outline}
%     \tableofcontents[currentsection,currentsubsection]
%   \end{frame}
% }

% If you wish to uncover everything in a step-wise fashion, uncomment
% the following command: 
%\beamerdefaultoverlayspecification{<+->}

\begin{document}

% Title Data. We keep it after \begin{document} 
% to enable editing text in BaKoMa TeX Word.

\title{Relational Calculus}

\date[WI 2016]{CSE 132A Winter 2016. 22 Jan 2016}

\subject{Relational Calculus} % Should be passed to PDF [YNI]

\begin{frame}
  \titlepage
\end{frame}

\section{Relational Calculus Overview}
\begin{frame}
  \frametitle{Existential and Universal Quantifiers}
  $\textcolor{red}{\exists b \in \text{Books} [...]}$\\
  There exists a book b where this condition holds.\\
  Analagous to "FROM Books b WHERE ..." in SQL.
  
  \hfill 
  
  $\textcolor{red}{\exists b \in \text{Books} [\forall c \in \text{Customers} [ \exists p \in \text{Purchases} [...] ] ]}$\\
  There exists a book, where for each customer, there exists a purchase where this condition holds.\\
  \hfill 
\end{frame}

\begin{frame}
  \frametitle{Tuples}
  Tuples are like rows in a database\\
  \hfill \\
  $\exists b \in \text{Books} [\textcolor{red}{b(\text{title})} ...]$\\
  \hfill \\
  $\exists b \in \text{Books} [\forall c \in \text{Customers} [ \exists p \in \text{Purchases} [ $
  $\textcolor{red}{c(\text{customerid}) ... p(\text{customerid}) ... p(\text{bookid}) ... b(\text{bookid})}] ] ]$\\
  \hfill \\
  Similar to b.title and p.price
\end{frame}

\begin{frame}
  \frametitle{SELECT Analogy}
  $\textcolor{red}{\{r:\text{bookid} \mid }\exists b \in \text{Books} [b(\text{title})
= \text{'Twilight'}$ \\
  $\textcolor{red}{\land r(\text{bookid}) = b(\text{bookid})}]\textcolor{red}{\}}$\\
  \hfill \\
  $\textcolor{red}{\{ r:\text{title} \mid }\exists b \in \text{Books} [\forall c \in
\text{Customers} [ \exists p \in \text{Purchases} [ $\\
  $c(\text{customerid}) = p(\text{customerid}) \land  p(\text{bookid}) = b(\text{bookid})$
\\
  $] ] \textcolor{red}{ \land b(\text{title}) = r(\text{title})]\}}$
\end{frame}

\begin{frame}
  \frametitle{Formal Definition}
  The general form of the relational calculus is:
  $$
  \{t | P(t)\}
  $$
  $t$ is a free variable. $P$ is the formula. A tuple variable is a \textit{free
variable} unless it is quantified by $\exists, \forall$. If quantified, it is \textit{bound
variable}. 
  $$
  t \in Purchases \wedge c \in Customers [t(custID)=c(custID)]
  $$
\end{frame}

\begin{frame}
  \frametitle{Atoms}
  \begin{itemize}
  \item $t \in R$, where $t$ is a tuple variable, $R$ is a relation
  \item $t(x) \Theta u(y)$, where $t, u$ are tuple variables, $x, y$ are the attributes. $\Theta$ is comparison operator ($<, \leq, =, \neq, \geq, >$).
  \item $t(x) \Theta c$, where $c$ is a constant. 
  \end{itemize}
\end{frame}

\begin{frame}
  \frametitle{Formulae}
  \begin{itemize}
  \item An atom $P$ is a formula
  \item $\neg P, P_1 \wedge P_2, P_1 \vee P_2$
  \item $P_1 \rightarrow P_2$
  \item $P(t)$ contains a free tuple $t$, and.
  \begin{align*}
  &\exits t \in R [P(t)]\\
  &\forall t \in R [P(t)]
  \end{align*}
  \end{itemize}
\end{frame}


\begin{frame}
  \frametitle{Equivalence}
  \begin{itemize}
  \item $P_1 \wedge P_2 \equiv \neg (\neg P_1 \vee \neg P_2)$
  \item $\forall t \in R[P(t)] \equiv \neg \exits t \in R [\neg P(t)$]
  \item $P_1 \rightarrow P_2 \equiv \neg P_1 \vee P_2$
  \end{itemize}
\end{frame}

\section{Examples}

\begin{frame}
  \frametitle{Example 1}
    \framesubtitle{SQL to Relational Calculus}
  Find books (show their titles) written by 'EDMUND MORGAN' since year 1990.\\
  \hfill \\  
  SELECT title\\
  FROM Books\\
  WHERE author = 'EDMUND MORGAN'\\
  AND year $>=$ 1990;\\
\end{frame}

\begin{frame}
  \frametitle{Example 1}
    \framesubtitle{SQL to Relational Calculus}
  SELECT title\\
  FROM Books\\
  WHERE author = 'EDMUND MORGAN'\\
  AND year $>=$ 1990;\\
  \hfill \\
  \textcolor{red}{$\{r:\text{title} \mid ... \}$}
\end{frame}

\begin{frame}
  \frametitle{Example 1}
    \framesubtitle{SQL to Relational Calculus}
  SELECT title\\
  FROM Books\\
  WHERE author = 'EDMUND MORGAN'\\
  AND year $>=$ 1990;\\
  \hfill \\
  $\{r:\text{title} \mid \textcolor{red}{\exists b \in \text{Books} [ ... ]} \}$
\end{frame}

\begin{frame}
  \frametitle{Example 1}
    \framesubtitle{SQL to Relational Calculus}
  SELECT title\\
  FROM Books\\
  WHERE author = 'EDMUND MORGAN'\\
  AND year $>=$ 1990;\\
  \hfill \\
  $\{r:\text{title} \mid \exists b \in \text{Books} [ \textcolor{red}{b(\text{author}) = \text{'EDMUND MORGAN'} \land b(\text{year}) \geq 1990} ] \}$\\
  DONE?
\end{frame}

\begin{frame}
  \frametitle{Example 1}
    \framesubtitle{SQL to Relational Calculus}
  SELECT title\\
  FROM Books\\
  WHERE author = 'EDMUND MORGAN'\\
  AND year $>=$ 1990;\\
  \hfill \\
  $\{r:\text{title} \mid \exists b \in \text{Books} [ b(\text{author}) = \text{'EDMUND MORGAN'} \land b(\text{year}) \geq 1990$\\
  $\textcolor{red}{\land r(\text{title}) = b(\text{title})}] \}$\\
\end{frame}

\begin{frame}
  \frametitle{Example 2}
    \framesubtitle{Relational Calculus to SQL}
  What are the titles of the newest books?\\
\end{frame}

\begin{frame}
  \frametitle{Example 2}
    \framesubtitle{Relational Calculus to SQL}
  What are the titles of the newest books?\\
  \hfill \\
  \textcolor{red}{$\{r:\text{title} \mid ... \}$}
\end{frame}
\begin{frame}
  \frametitle{Example 2}
    \framesubtitle{Relational Calculus to SQL}
  What are the titles of the newest books?\\
  \hfill \\
  $\{r:\text{title} \mid \textcolor{red}{\exists b \in \text{Books} [...]} \}$
\end{frame}
\begin{frame}
  \frametitle{Example 2}
    \framesubtitle{Relational Calculus to SQL}
  What are the titles of the newest books?\\
  \hfill \\
  $\{r:\text{title} \mid \exists b \in \text{Books} [ \textcolor{red}{\forall o \in \text{Books} [...]} ] \}$
\end{frame}
\begin{frame}
  \frametitle{Example 2}
    \framesubtitle{Relational Calculus to SQL}
  What are the titles of the newest books?\\
  \hfill \\
  $\{r:\text{title} \mid \exists b \in \text{Books} [ \forall o \in \text{Books} [ \textcolor{red}{b(\text{year}) \geq o(\text{year})}]] \}$\\
  DONE?
\end{frame}
\begin{frame}
  \frametitle{Example 2}
    \framesubtitle{Relational Calculus to SQL}
  What are the titles of the newest books?\\
  \hfill \\
  $\{r:\text{title} \mid \exists b \in \text{Books} [ \forall o \in \text{Books} [ b(\text{year}) \geq o(\text{year})]$\\
  $\land \textcolor{red}{b(\text{title}) = r(\text{title})}] \}$
\end{frame}
\begin{frame}
  \frametitle{Example 2}
    \framesubtitle{Relational Calculus to SQL}
  What are the titles of the newest books?\\
  \hfill \\
  $\{r:\text{title} \mid \exists b \in \text{Books} [ \textcolor{red}{\neg\exists} o \in \text{Books} [ b(\text{year}) \textcolor{red}{<} o(\text{year})]$\\
  $\land b(\text{title}) = r(\text{title})] \}$
\end{frame}
\begin{frame}
  \frametitle{Example 2}
    \framesubtitle{Relational Calculus to SQL}
  What are the titles of the newest books?\\
  \hfill \\
  $\{r:\text{title} \mid \exists b \in \text{Books} [ \neg\exists o \in \text{Books} [ b(\text{year}) < o(\text{year})]$\\
  $\land b(\text{title}) = r(\text{title})] \}$\\
  \hfill \\
  SELECT b.title\\
  FROM Books b\\
  WHERE ...
\end{frame}
\begin{frame}
  \frametitle{Example 2}
    \framesubtitle{Relational Calculus to SQL}
  What are the titles of the newest books?\\
  \hfill \\
  $\{r:\text{title} \mid \exists b \in \text{Books} [ \neg\exists o \in \text{Books} [ b(\text{year}) < o(\text{year})]$\\
  $\land b(\text{title}) = r(\text{title})] \}$\\
  \hfill \\
  SELECT b.title\\
  FROM Books b\\
  WHERE NOT EXISTS( SELECT\\
  FROM Books o \\
  WHERE b.year $<$ o.year);
\end{frame}
\begin{frame}
  \frametitle{Example 2}
    \framesubtitle{Relational Calculus to SQL}
  What are the titles of the newest books?\\
  \hfill \\
  $\{r:\text{title} \mid \exists b \in \text{Books} [ \neg\exists o \in \text{Books} [ b(\text{year}) < o(\text{year})]$\\
  $\land b(\text{title}) = r(\text{title})] \}$\\
  \hfill \\
  SELECT b.title\\
  FROM Books b\\
  WHERE NOT EXISTS( SELECT \\
  FROM Books o\\
  WHERE b.year $<$ o.year);
\end{frame}

\end{document}

